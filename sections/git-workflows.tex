\documentclass[../main.tex]{subfiles}
\begin{document}

\section{Git 工作流}

Git作为一种强大的分布式版本控制系统,如果使用时的工作流程分支策略不当,便会导致版本控制的混乱,从而影响团队的协作效率。
因此,Git 工作流是Git的一个重要特性。
下面将介绍Git的几种常见工作流及其优缺点。

\subsection{传统工作流}

经典的 Git 工作流采用的是以下三种常见的分支策略:
Gitflow、GitHub Flow、GitLab Flow。

Gitflow是最早提出的分支模型之一,由Vincent Driessen 在 nvie 上首次提出\cite{gitflow},
主要针对采用集中式工作流的项目。
Gitflow 工作流同时使用两种长期分支:main 和 develop。
main 分支用于存放稳定版本,develop 分支用于存放开发版本。
每个新特性都在自己的功能分支上开发,开发完成后合并到 develop 分支。功能分支不与 main 分支直接交互。
当 develop 分支上的新功能开发完成后,先从 develop 分支分离出 release 分支。
当 release 分支准备好发布时,最终合并到 main 分支并打上标签。

Gitflow 工作流的优点是分支模型清晰,适合基于版本发布的项目。
但是 Gitflow 需要同时维护两个长期分支,相对复杂。且对于持续发布的项目,main分支和develop分支的差异并不大。

GitHubFlow 是 GitHub 推荐的轻量级的工作流\cite{githubflow}。

在GitHubFlow中,只有一个长期分支 main。
开发时,从 main 分支分离出新的功能分支。
待功能分支开发完成后,需要提交一个 Pull Request,由其他人进行代码审查。
当代码审查通过后,功能分支便可合并到 main 分支。

GitHubFlow 的优点是分支模型简单,不仅仅可供软件开发使用,还可供文档类项目使用。例如本文的编写便是使用GitHubFlow进行协作的。
此外,GitHubFlow 相较于Gitflow,更加适应社区型的分布式协作。
但是 GitHubFlow 也有一些缺点,例如只有一条长期分支,如果由于一些原因不能立刻发布,就会导致发布版本落后于 main 分支版本。

GitLab Flow 结合了以上两种分支模型的特点。\cite{gitlabflow}例如,GitLab Flow支持类似于 Pull Request 的 Merge Request 机制。与此同时,也允许多个长期分支。

GitLab Flow 采取了“上游优先”(upstream first)策略,例如 main 分支作为上游开发分支,pre-production 分支作为下游预发布分支,production 分支作为下游发布分支。
进行更改时,一般而言必须先在 main 分支上进行更改,然后再合并到 pre-production 分支和 production 分支,也就是从上游到下游。

\subsection{主干-分支工作流}

主干-分支模型,或基于主干的工作流(trunk based),是一种更加简化,更加灵活的分支模型。
许多大型的开发组织,例如谷歌和脸书,都采用了这种分支模型。\cite{trunkbased}
在基于主干的分支模型下,一切发布和修复都是围绕着主干进行的。开发者不断地向主干提交小的、频繁的更新,而不需要像Gitflow那样待功能分支完成后在合并到主干。

基于主干的分支模型可以带来很多好处。例如使得持续集成成为可能,更加利于进行自动化的构建和测试,减少代码集成中产生的摩擦,保证持续的代码审查和发布等等。
\subsection{分布式工作流程}

Git 作为分布式版本管理系统,开发者间的协作方式是灵活多样的。
根据每个开发者角色的不同,可以产生不同的工作流程。
常见的的有集中式工作流、集成管理者工作流、主管与副主管工作流。

集中式工作流是一种简单的工作流。一个中心仓库接受所有开发者的代码并与中心仓库同步。
如果两个开发者同时克隆了中心仓库的代码并做了一些修改,那么只有第一个开发者能提交修改,第二个开发者提交前必须先将第一个人的工作合并。
这种模式的使用非常广泛,小团队和大团队都可以采取这种方式进行协作。

集成管理者工作流是一种更加复杂的工作流,每个开发者拥有自己仓库的写权限和所有仓库的读权限。
往往有存在一个中心的“权威”仓库,所有开发者克隆这个仓库,在自己的仓库中进行工作,然后请求将自己的工作合并进这个仓库。
这是GitHub和GitLab等工具的常用工作流,可以方便的实现持续性的社区协作。

主管与副主管工作流一般只有超大型项目(例如Linux内核)才会采用。一位主管负责管理统筹整个项目,维护一个参考仓库。多个副主管分别负责项目中的特定部分,维护自己的分支以便合并进主管的仓库。
这种方式并不十分常见,但是在大型项目中,这种方式可以保证项目的稳定性。

\end{document}

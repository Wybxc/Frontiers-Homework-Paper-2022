\documentclass[../main.tex]{subfiles}
\begin{document}

\section{基于 GitHub 的现代开源项目}

GitHub 是一个在线的代码托管和版本管理平台,因其开放的源代码托管能力和强大的社区协作功能,已经成为了全球最大的开源项目托管平台,目前已有超过六千万用户。

\subsection{GitHub}

GitHub于2008年正式上线后,发展十分迅速,2010年便拥有了多达百万的 repository \cite{github_million},2013年这个数字便达到了一千万\cite{github_ten_million}。

GitHub 并不仅为开源项目提供服务,在用户创建 repository 时,可以指定 repository 的访问权限,例如公开、私有以及使用何种开源协议等。这使得 GitHub 也可以用于企业内部的代码管理。
除了大量个人开发者外,许多企业也是 GitHub 的用户。GitHub 的用户不仅有程序员,还有很多非程序员的用户,例如设计师、文档编辑者等等。这些用户可以在 GitHub 上创建自己的 repository,然后与其他用户进行交流和协作,这篇文章便使用的是 GitHub 管理 \LaTeX 源文件并进行协作。

\subsection{开源项目的性质与规模}

开源软件是开放源代码并允许他人学习、修改、分发的软件。
在二十世纪九十年代末,随着 Linux 受到大众认可以及 Netscape 浏览器公开源代码,开源的理念逐渐变得流行\cite{opensource}。
根据项目所有者的身份,现代开源项目主要分为个人项目、社区项目和公司项目等。

个人项目的项目所有者与主要维护者为一个人,其他人可以通过 Pull Request 参与。其规模一般较小。

社区项目所有者为一个社区,主要维护者为一个或几个人。社区内的非主要维护者的成员可以通过 Pull Request 参与。其规模一般较大,例如 Linux 内核、Apache HTTP Server 等。

公司项目所有者为一个公司,主要维护者为公司中的一个团队,不接受或有限的接受外部人员的 Pull Request。公司项目的规模一般较大,例如 Google 的 Android 系统。

同时,项目规模也对 Pull Request 的接受程度有影响,小型项目一般比较宽松,中大型项目要么必须通过一套规定流程后再由维护者决定,要么不接受。
不过,目前越来越多的公司项目也在利用类似社区项目的运营模式,如同社区项目般接受 Pull Request。

开源运动极大地促进了计算机科学的发展,个人和公司都能在向开源社区做贡献的同时受益其中。

\subsection{社区协作与持续集成}

开源运动的成功离不开良好的社区参与机制,这也是 GitHub 的核心功能之一。GitHub 通过 Pull Request 等机制实现了社区协作,这一机制允许任何人向项目提交代码,项目维护者可以选择接受、拒绝、改动,也可与贡献者交流。

在有广泛社区参与的情况下,保证不同贡献者提交的代码风格统一、质量合格是协作的重要一环。持续集成(Continuous integration,CI)通过自动化的构建、测试,
可以确保不同贡献者的提交经过统一的测试流程,保证代码质量。

GitHub Actions 是 GitHub 提供的持续集成服务,可以通过简单的配置文件实现自动化构建、测试等功能。许多GitHub项目采用了基于主干的工作流,
为贡献者提出的 Pull Request 自动运行测试,便于维护者快速判断提交的代码是否合格。

\end{document}

\documentclass[../main.tex]{subfiles}
\begin{document}

\section{基于 GitHub 的现代开源项目}

GitHub 是一个在线的代码托管和版本管理平台,因其开放的源代码托管能力和强大的社区协作功能,已经成为了全球最大的开源项目托管平台。目前,GitHub 已有超过六千万的用户。

\subsection{GitHub}

GitHub于2008年正式上线后,发展十分迅速,2010年便拥有了多达百万的 repository \cite{github_million},2013年这个数字便达到了一千万\cite{github_ten_million}。

GitHub 提供了非常丰富的服务,不仅有完善的版本控制、代码托管服务,还提供了 star, follow, 评论等一系列社交功能。
GitHub 并不仅为开源项目提供服务,在用户创建 repository 时,可以指定 repository 的访问权限,例如公开、私有以及使用何种开源协议等。这使得 GitHub 也可以用于企业内部的代码管理。
除了大量个人开发者外,许多企业也是 GitHub 的用户。例如微软在收购 GitHub 之前,便已经利用 GitHub 托管了大量开源项目、开发工具及其文档。

GitHub 的用户不仅有程序员,还有很多非程序员的用户,例如设计师、文档编辑等等。这些用户可以在 GitHub 上创建自己的 repository,然后与其他用户进行交流和协作,这篇文章便使用的是 GitHub 管理 \LaTeX 源文件并进行协作。

\subsection{开源项目的性质与规模}

开源软件是开放源代码并允许他人学习、修改、分发的软件。开源理念的历史几乎和软件开发一样长,
在二十世纪九十年代末,随着 Linux 受到大众认可以及 Netscape 浏览器公开源代码,开源的理念逐渐变得流行\cite{opensource}。
根据项目所有者的身份,现代开源项目主要分为个人项目、社区项目和公司项目等。

个人项目指项目所有者与主要维护者为一个人,其他人可以通过 Pull Request 参与。个人项目的规模一般较小,例如个人博客、个人网站等。

社区项目指项目所有者为一个社区,主要维护者为一个或几个人,社区内的非主要维护者的成员可以通过 Pull Request 参与。社区项目的规模一般较大,例如 Linux 内核、Apache HTTP Server 等。

公司项目指项目所有者为一个公司,主要维护者为公司中的一个团队,不接受或有限的接受外部人员的 Pull Request。公司项目的规模一般较大,例如 Google 的 Android 系统、Facebook 的 React 等。

同时,项目规模也对 Pull Request 的接受程度有影响,小型项目一般比较宽松,中大型项目要么必须通过一套规定流程后再由维护者决定,要么(例如一些公司项目)不接受。
不过,目前越来越多的公司项目也在利用类似社区项目的运营模式,如同社区项目般接受 Pull Request,例如 Google 的 TensorFlow 等。

开源运动极大地促进了计算机科学的发展,个人开发者和商业公司都能在向开源社区做贡献的同时从中获得收益。

\subsection{社区协作与持续集成}

开源运动的成功离不开一个良好的社区参与机制,这也是 GitHub 的核心功能之一。GitHub 通过已经多次提到的 Pull Request 机制实现了社区协作,Pull Request 机制允许任何人向项目提交代码,项目维护者可以选择接受或拒绝这些代码。

在有广泛社区参与的情况下,保证不同贡献者提交的代码风格统一、质量合格是协作的重要一环。持续集成(Continuous integration,CI)通过自动化的构建、测试,
可以确保不同贡献者的提交经过统一的测试流程,保证代码质量。

GitHub Actions 是 GitHub 提供的持续集成服务,可以通过简单的配置文件实现自动化构建、测试等功能。许多GitHub项目采用了基于主干的工作流,
为贡献者提出的 Pull Request 自动运行测试,便于维护者快速判断提交的代码是否合格。

\end{document}

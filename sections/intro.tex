\documentclass[../main.tex]{subfiles}
\begin{document}

\section{引言}

版本控制是一种记录一个或若干文件内容变化,以便将来查阅特定版本修订情况的系统
\cite{progit}。在软件开发中,为了更好地管理软件的功能变更,协调多人合作,
往往会使用版本控制系统。

Git 是一个开源的分布式版本控制系统,由 Linus Torvalds 于 2005 年开发,
最初用于管理 Linux 内核的源码。由于其结构简单而功能强大,
如今被广泛应用于开源项目的版本控制中。
Git 的设计理念遵循 Unix 哲学,提倡使用小而简单的工具来完成单一的任务,
以工具的组合来完成复杂的任务。

Git 的这一设计理念使得 git 的工具集可以有多样的组合方式,
从而可以适应不同的应用场景。对于一个特定的项目,开发者需要根据项目的特点,
选定一种恰当的工具集组合方式,即工作流,并与所有协作者约定好使用的工作流.
在不同的开源项目中,根据性质、规模、托管平台、持续集成等因素的不同,
使用的工作流也有所不同。

GitHub 是一个基于 git 的面向开源及私有软件项目的托管平台,
是目前全球最大的开源社区之一,构建了一套完整的开源软件开发模式。
以 GitHub 为代表的代码托管平台,在 git 的基础上,提供了一系列额外的协作功能,
如 issue、pull request、fork、release、action 等,
这些功能的加入,让现代的开源软件开发流程有别于早期只使用 git 本身的流程,
发展出新的工作流。

本文将对不同的 git 工作流进行对比,分析其在现代开源软件开发中的应用场景,
并以实际项目为例,介绍这些工作流在实际开发中的使用。

\end{document}

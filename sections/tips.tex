\documentclass[../main.tex]{subfiles}
\begin{document}

\section{Tips: 基于Git/Github的项目开发中的工作流选择}

不同的项目,其性质和规模不同、集成时产生冲突摩擦的程度不同、开发者之间的协作关系也各不相同,所以并不存在某种统一的“最佳工作流”,而需要项目管理者结合实际情况进行选择。

\subsection{不同项目的特点分析}

首先,项目的性质决定了参与者的协作关系,这主要决定了项目协作的工作流程。在某些小型项目中,所有参与者都属于项目维护者,所有成员均可对共享的仓库直接进行更改。另一些项目中,有明确的主要维护者,即项目负责人,其他开发者的版本提交需经过主要维护者的审查。而更大型的项目,特别是社区开源项目中,主要维护者为项目的核心团队,其下管理着数量庞大的普通开发者。

项目的内容和规模不同,也使得集成时产生冲突的程度不同。这主要决定了项目的分支管理策略。代码类项目集成时,可能面临很多潜在的冲突,需要进行严格测试;主分支的更改,也会给分支合并时带来很多不便。相比而言,其他一些项目如文档类的项目,集成时的冲突便小很多。

另一方面,一些项目有明显分块结构,或有不同的相对独立的功能。正如之前所提到的,这些项目可以针对某个模块和功能,创建特性分支,完成后合入主分支。这样在合并时,产生的摩擦也相对较少。

\subsection{选择工作流程与协作方式}

对于所有参与者平等参与开发的小型项目,可以采用集中式工作流,这样的协作方式较为简单,避免了很多冗余流程。有明确的项目主导角色时,如果项目规模不大,同样可以采用这样的工作流,项目管理者同时作为普通开发者参与版本提交。

有明确主导角色且项目具有一定规模时,可以采用集成管理者工作流。缺点是流程较为繁琐,而且需要保证项目管理者及时地拉取其他开发者的版本更新。

对于大型的、社区协作开发的项目,可以采用集成管理者工作流,由核心团队担任项目管理者角色;如果项目规模特别庞大,需要保证项目稳定性,也可以像linux一样,采用主管与副主管工作流,形成多级管理。

\subsection{分支策略的选择}

是否需要设置develop长期分支,是Gitflow和GitHubFlow的区别之一。小型项目最好采用轻量级的管理策略,可以通过GitHub的PR机制进行分支维护,不需要长期develop分支。需要持续发布稳定版本的大型项目,可以采用Gitflow分支策略。

有明显分块结构或功能划分的项目,可以创建多个特性分支。如果各个特性之间互不影响,可以采用类似Gitflow中特性分支的管理策略;若需要维护代码稳定性的同时持续集成发布,可以考虑主干-分支工作流,通过短周期特性分支进行小而频繁的主干更新。但需要注意,在开发的最初阶段,需要先形成稳定的项目框架,此时主干变化较为剧烈,不宜进行分支。

\end{document}
